\def\classopts{11pt}
\input ../beamertemplate.tex

\definecolor{good}{cmyk}{0.50, 0.00, 0.70, 0.20}
\definecolor{bad}{cmyk}{0.00, 0.50, 1.00, 0.50}
\definecolor{ok}{cmyk}{0.50, 0.10, 1.00, 0.50}
\definecolor{check}{cmyk}{0.90, 0.20, 0.20, 0.40}

\mode<handout>{%
    \nofiles % do not overwrite 
    \usetheme{workshopdtu}
}
\mode<presentation>
{
    \usetheme{workshopdtu}
}

\usepackage{listings}
\usepackage{fancyvrb}

% Change logo to cecam
%\logo{\includegraphics[width=\logosize]{SIESTA_small}}

% Explicitly set the compatibility version
\pgfplotsset{compat=1.15}

\author{Nick R. Papior}

\usetikzlibrary{%
    fadings,calc,arrows,
    positioning,
    decorations,
    decorations.markings,
    decorations.pathreplacing,
    decorations.text,
    matrix,
}

\tikzset{
    ->-/.style={postaction={decorate},decoration={%
            markings,mark=at position #1 with {\arrow[line width=2pt]{stealth}}%
        }%
    },%
    ->-/.default=.5,%
    -<-/.style={postaction={decorate},decoration={%
            markings,mark=at position #1 with {\arrowreversed[line width=2pt]{stealth}}%
        }%
    },%
    -<-/.default=.5,%
}
\usepgfplotslibrary{groupplots}

\usepackage{algorithm}
\usepackage{algpseudocode}

\usepackage{biblatex}
\bibliography{refs}


\def\tsiesta{\texttt{transiesta}}
\def\siesta{\texttt{siesta}}
\def\tbt{\texttt{tbtrans}}
\def\fdf#1{\texttt{#1}}
\def\mcite#1{{\small \citeauthor{#1}, DOI: \citefield{#1}{doi}}}

\newwrite\foundfile
\immediate\openout\foundfile=\jobname.foundfiles
\newwrite\nonfoundfile
\immediate\openout\nonfoundfile=\jobname.nonfoundfiles



\makeatletter
\newcommand\incg[2][align=c]{%
    \bgroup%
    \let\tmp@file\relax%
    \incg@check{#2}{png}%
    \ifx\tmp@file\relax%
      \incg@check{#2}{pdf}%
    \fi%
    \ifx\tmp@file\relax%
      \incg@check{#2}{jpg}%
    \fi%
    \ifx\tmp@file\relax%
      \incg@check{#2}{jpeg}%
    \fi%
    \ifx\tmp@file\relax%
      \immediate\write\nonfoundfile{#2}%
      \rule{3cm}{4cm}%
    \else%
      \immediate\write\foundfile{\tmp@file}%
      \includegraphics[#1]{\tmp@file}%
    \fi%
    \let\tmp@file\relax%
    \egroup%
}
\newcommand\incg@check[2]{%
    \IfFileExists{#1.#2}{%
        % It does
        \edef\tmp@file{#1.#2}%
    }{}%
    \ifx\tmp@file\relax%
      \IfFileExists{fig/#1.#2}{%
          % It does
          \edef\tmp@file{fig/#1.#2}%
      }{}%
    \fi%
    \ifx\tmp@file\relax%
      \IfFileExists{people/#1.#2}{%
          % It does
          \edef\tmp@file{people/#1.#2}%
      }{}%
    \fi%
}
\makeatother


% Define all variables used in this paper
\newcommand\G{\mathbf{G}}
\newcommand\Spec{\mathcal{A}}
\newcommand\Gr{\mathbf{G}^r}
\newcommand\Ga{\mathbf{G}^a}
\newcommand\HH{\mathbf{H}}
\newcommand\VV{\mathbf{V}}
\newcommand\DM{\boldsymbol{\rho}}
\newcommand\DE{\DM_\Eq}
\newcommand\DN{\DM_\Neq}
\newcommand\ncor{\boldsymbol{\Delta}}
\newcommand\SO{\mathbf{S}}
\newcommand\SE{\boldsymbol{\Sigma}}
\newcommand\Scat{\boldsymbol{\Gamma}}
\newcommand\kk{\mathbf{k}}
\newcommand\RR{\mathbf{R}}
\newcommand\rr{\mathbf{r}}
\newcommand\im{\mathfrak{i}}
\newcommand\ID{\mathbf{I}}
\newcommand\sumE{\sum^{\mathfrak{E}}}
\newcommand\NE{\mathfrak{E}}
\newcommand\idxE{\mathfrak{e}}
\newcommand\sumU{\sum^{\NU}}
\newcommand\NU{N_\mu}
\newcommand\idxU{\mu}
\newcommand\cd{\!\dd}
\newcommand\Cd{\!\!\dd}
\newcommand\dd{\mathrm{d}}
\newcommand\E{\epsilon}
\newcommand\JJ{\mathcal{J}}
\newcommand\BZ{\mathrm{BZ}}
\newcommand\Eq{\mathrm{eq}}
\newcommand\Neq{\mathrm{neq}}
\newcommand\varneq{\boldsymbol{\beta}}
\DeclareMathOperator\Var{Var}
\DeclareMathOperator\Tr{Tr}
\newcommand\dEBZ{\!\!\!\!\!\!\!\!}
\newcommand\dE{\!\!\!\!}
\newcommand\DOS{\mathrm{DOS}}
\newcommand\T{\mathcal{T}}
\newcommand\RE{\mathcal{R}}
\def\mto{\rightarrow}
\newcommand\mrc[2]{%
    \bgroup%
    \{#1,#2\}%
    \egroup%
}
\newcommand\eikr{e^{-\im\kk\cdot\RR}}

\def\etal{\emph{et.al.}}
\newcommand\doi[1]{\href{https://dx.doi.org/#1}{#1}}


\colorlet{cm1}{green!80!black}
\colorlet{cm2}{red!80!black}
\tikzset{
    s >/.style={shorten >=#1},
    s >/.default=4pt,
    s </.style={shorten <=#1},
    s </.default=4pt,
    s <>/.style={s >=#1,s <=#1},
    s <>/.default=4pt,
    >=latex,
    font=\footnotesize,
    block/.style={
        local bounding box=localbb,
        execute at end scope={
            \node[rounded corners=5pt,
            inner sep=5pt,
            fit=(localbb),draw,#1] {};
        },
    },
    fdf/.style={
        font=\ttfamily\footnotesize,
        anchor=west,
    },
    connect/.style={
        >=latex,
        ->,
        shorten >=4pt,
        shorten <=1pt,
    },
    my mark/.style={
        rectangle,rounded corners=2pt,very thick,
        inner sep=2pt,
    },
    lmark/.style={
        my mark,draw=cm1,%green!80!black,
    },
    rmark/.style={
        my mark,draw=red!80!black,densely dotted,
    },
    iiind/.style={xshift=6ex},
    iind/.style={xshift=4ex},
    ind/.style={xshift=2ex},
    % Matrix opts
    row sep=.4ex,
    every node/.append style={
        inner sep=.25ex,outer sep=.25ex,
        text height=1.25ex,
    },
    ampersand replacement=\&,
}


\def\matsize{.8cm}
\def\Full{\begin{tikzpicture}
      \fill (0,0) rectangle (\matsize,\matsize);
    \end{tikzpicture}
}
\def\Left{\begin{tikzpicture}
      \draw (0,0) rectangle (\matsize,\matsize);
      \fill (0,\matsize) rectangle ++(.3*\matsize,-.3*\matsize);
    \end{tikzpicture}
}
\def\FullLeft{\begin{tikzpicture}
      \draw (0,0) rectangle (\matsize,\matsize);
      \fill (0,0) rectangle (.3*\matsize,\matsize);
    \end{tikzpicture}
}

\def\drawcube[#1]{%
    \path[#1] (0,0,0) -- (0,0,1);
    \path[#1] (0,0,0) -- (1,0,0);
    \path[#1] (0,0,0) -- (0,1,0);
    \path[#1] (1,0,0) -- (1,1,0);
    \path[#1] (1,0,0) -- (1,0,1);
    \path[#1] (0,1,0) -- (1,1,0);
    \path[#1] (0,1,0) -- (0,1,1);
    \path[#1] (0,0,1) -- (1,0,1);
    \path[#1] (0,0,1) -- (0,1,1);
    \path[#1] (1,1,0) -- (1,1,1);
    \path[#1] (1,1,1) -- (0,1,1);
    \path[#1] (1,1,1) -- (1,0,1);
}

