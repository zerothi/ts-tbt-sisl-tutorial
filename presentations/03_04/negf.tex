\section{Non-equilibrium Green function}

\begin{framenologo}
  \frametitle{Non-equilibrium Green function}
  \tableofcontents[currentsection]
\end{framenologo}

\subsection{Density matrix}

\begin{frame}
  \frametitle{Density matrix}
  \framesubtitle{Recollection of basic equations}

  \begin{block}{Basic equations for Green function techniques}
    \vskip-2ex
\begin{align*}
  \G_\kk(\E)&=\big[(\E+\im\eta)\SO_\kk - \HH_\kk - \sum_\idxE\SE_{\idxE,\kk}(\E-\mu_\idxE)\big]^{-1}
  \\
  \Scat_{\idxE,\kk}(\E) &=i\big(\SE_{\idxE,\kk}(\E-\mu_\idxE)-\SE^\dagger_{\idxE,\kk}(\E-\mu_\idxE)\big)
  \\
  \Spec_{\idxE,\kk}(\E) &=\G_\kk(\E)\Scat_{\idxE,\kk}(\E)\G_\kk^\dagger(\E)
  % \\
  % \DM & =\frac1{2\pi}
  % \iint_\BZ\dEBZ\cd \kk \dd\E\, \sum_\idxE\Spec_{\idxE,\kk}(\E) n_{F,\idxE}(\E)
  % \eikr,
\end{align*}
  \end{block}

\end{frame}

\begin{frame}
  \frametitle{Density matrix}
  \framesubtitle{Equilibrium}

  \begin{itemize}
    \item All electrodes have same Fermi-distribution and here the density is easily
    calculated using \emph{only} the Green function
  \end{itemize}

  \begin{block}<+->{Equilibrium density matrix $V=0$}
    \vskip -2ex
    \begin{align*}
      \DM &= \frac1{2\pi}\iint_\BZ\dEBZ\cd \kk \dd\E\, %
      \G_\kk \sum_\idxE\Scat_{\idxE,\kk} \G^\dagger_\kk\eikr n_F(\E) %
      \\
      \DE &= \frac i{2\pi}\iint_\BZ\dEBZ\cd \kk\dd\E\, %
      \G_\kk \big[\SE_{\kk}-\SE^\dagger_{\kk} \big]
      \G^\dagger_\kk\eikr n_F(\E) & \SE_\kk = \sum_\idxE \SE_{\idxE,\kk} %
      \\
      \DE & = \frac i{2\pi}\iint_\BZ\dEBZ\cd \kk\dd\E\, %
      \G_\kk \left[%
        \G^{\dagger,-1}_\kk - \G^{-1}_\kk + 2i\eta\SO_\kk %
      \right]\G^\dagger_\kk\eikr n_F(\E)
      \\
      \DE & = \frac i{2\pi}\iint_\BZ\dEBZ\cd \kk\dd\E%
      \left[\G_\kk  - \G^\dagger_\kk + 2i\eta\G_\kk\SO_\kk\G^\dagger_\kk %
      \right]\eikr n_F(\E)
      \\
      &\approxeq \frac i{2\pi}\iint_\BZ\dEBZ\cd \kk\dd\E%
      \left[\G_\kk  - \G^\dagger_\kk \right]\eikr n_F(\E)
    \end{align*}
  \end{block}
  
\end{frame}


\begin{frame}
  \frametitle{Density matrix}
  \framesubtitle{Non-equilibrium}

  \begin{itemize}
    \item Electrodes \emph{may} have different Fermi-distribution (at least two different)
  \end{itemize}

  \begin{block}<+->{Non-equilibrium density matrix $V\neq0$}
    \begin{itemize}
      \item Non-equilibrium density may conveniently be split into an equilibrium part and
      a non-equilibrium ``correction''
    \end{itemize}
    \begin{align*}
      \DM &= \frac1{2\pi}\iint_\BZ\dEBZ\cd \kk \dd\E\, %
      \G_\kk \sum_\idxE\Scat_{\idxE,\kk} \G^\dagger_\kk\eikr n_{F,\idxE}(\E) %
      \\
      \DM=\DE^\idxE+\DN^\idxE &=\DE^\idxE+
      \frac1{2\pi}\sum_{\idxE'\neq\idxE}\iint_\BZ\dEBZ\cd \kk\dd\E\, %
      \G_\kk \Scat_{\idxE',\kk}
      \G^\dagger_\kk\eikr\big[n_{F,\idxE'}(\E)-n_{F,\idxE}(\E)\big]
      \uncover<3->{
      \\
      \DM=\DE^\varsigma+\DN^\varsigma &=\DE^\varsigma+
      \frac1{2\pi}\sum_{\idxE'\neq\varsigma}\iint_\BZ\dEBZ\cd \kk\dd\E\, %
      \G_\kk \Scat_{\idxE',\kk}
      \G^\dagger_\kk\eikr\big[n_{F,\idxE'}(\E)-n_{F,\varsigma}(\E)\big]
      }
    \end{align*}
    \begin{itemize}
      \item<+-> %
      Note that if two electrodes have the same Fermi-distribution we have
      $n_{F,\idxE}=n_{F,\idxE'}$

      \item<+-> %
      Enables the reduction of ($N_\idxE$) different equations to the number of different
      chemical potentials ($N_\varsigma$)

    \end{itemize}

  \end{block}
  
\end{frame}


\subsection{Numeric integration}

\begin{frame}
  \frametitle{Numeric integration}
  \framesubtitle{Equilibrium}

  Calculation of $\DE^\varsigma$:

  \begin{center}
    \def\eta{0.1}%
    \def\radius{3.25}%
    \def\lineS{-1}%
    \def\poles{4}%
    \def\poleSep{.25}%
    % Calculate alpha angle
    \pgfmathparse{\poleSep*(\poles+.5)/\radius}%
    \edef\betaA{\pgfmathresult}%
    \pgfmathparse{atan(\betaA)}%
    \edef\alphaA{\pgfmathresult}%
    \pgfmathparse{asin(\betaA)}%
    \edef\betaA{\pgfmathresult}%
    \begin{tikzpicture}[scale=.75]
      
      % The axes
      \begin{scope}[draw=gray!80!black,thick,->]
        \draw (-2*\radius+\lineS-.5,0) -- (\radius+1.5,0) node[text=black,below] {$E$};
        \draw (0,0) -- (0,\radius+.5) node[text=black,left] {$\Im$};
      \end{scope}
      \node[below] at (0,0) {$\mu$};
      
      % The specific coordinates on the path
      \coordinate (EB) at (-2*\radius+\lineS,\eta);
      \coordinate (C-mid) at ({-\radius+\lineS-sin(\alphaA)*\radius},{cos(\alphaA)*\radius});
      \coordinate (C-end) at (\lineS,{\poleSep*(\poles+.5)});
      \coordinate (L-end) at (\radius,{\poleSep*(\poles+.5)});
      \coordinate (L-end-end) at (\radius+1,{\poleSep*(\poles+.5)});
      \coordinate (real-L-end) at (\radius,\eta);
      \coordinate (real-L-end-end) at (\radius+1,\eta);
      
      \begin{scope}[thick]
        
        % The path (we draw it backwards)
        \draw[->-=.3,very thick] (L-end) -- node[above right] 
        {$\mathcal L$} (C-end);
        \draw[->-=.333,->-=.666,very thick] (C-end) to[out=90+\betaA,in=\alphaA] (C-mid)
        node[above] 
        {$\mathcal C$}
        to[out=180+\alphaA,in=90] (EB);
        \draw[->-=.25,->-=.75] (EB) -- (real-L-end) node[above left] {$\mathcal R$};

        % draw the continued lines
        \draw[densely dotted] (real-L-end) -- (real-L-end-end);
        \draw[densely dotted] (L-end) -- (L-end-end);

      \end{scope}
      
      % Draw the poles
      \foreach \pole in {1,...,14} {
          \ifnum\pole>\poles
          \draw (0,\pole*\poleSep) circle (2pt);
          \else
          \fill (0,\pole*\poleSep) circle (2pt);        
          \fi
      }
      \node[left,anchor=east] at (0,{\poleSep*(\poles/2+.5)}) {$z_\nu$};

      % correct size
      \path[use as bounding box] (-8,-.5) rectangle ++(13,4.5);

      \draw[densely dotted] (real-L-end-end) to[out=0,in=0] (L-end-end);
      \draw[densely dotted,thick] (EB) -- ++(-.5,0);

      \uncover<2->{
          \begin{scope}
            \def\muS{.5}%
            \node[below] at (\muS,0) {$\mu'$};

            
      % The specific coordinates on the path
      \coordinate (EB) at (-2*\radius+\lineS+\muS,\eta);
      \coordinate (C-mid) at ({-\radius+\lineS-sin(\alphaA)*\radius+\muS},{cos(\alphaA)*\radius});
      \coordinate (C-end) at (\lineS+\muS,{\poleSep*(\poles+.5)});
      \coordinate (L-end) at (\radius+\muS,{\poleSep*(\poles+.5)});
      \coordinate (L-end-end) at (\radius+1+\muS,{\poleSep*(\poles+.5)});
      \coordinate (real-L-end) at (\radius+\muS,\eta);
      \coordinate (real-L-end-end) at (\radius+1+\muS,\eta);
      
      \begin{scope}[thick,color=red]
        
        % The path (we draw it backwards)
        \draw[->-=.3,very thick] (L-end) -- node[above right] 
        {$\mathcal L$} (C-end);
        \draw[->-=.333,->-=.666,very thick] (C-end) to[out=90+\betaA,in=\alphaA] (C-mid)
        node[above] 
        {$\mathcal C$}
        to[out=180+\alphaA,in=90] (EB);
        \draw[->-=.25,->-=.75] (EB) -- (real-L-end) node[above left] {$\mathcal R$};

        % draw the continued lines
        \draw[densely dotted] (real-L-end) -- (real-L-end-end);
        \draw[densely dotted] (L-end) -- (L-end-end);

      \end{scope}
      
      % Draw the poles
      \foreach \pole in {1,...,14} {
          \ifnum\pole>\poles
          \draw (\muS,\pole*\poleSep) circle (2pt);
          \else
          \fill (\muS,\pole*\poleSep) circle (2pt);        
          \fi
      }

      \draw[densely dotted] (real-L-end-end) to[out=0,in=0] (L-end-end);
      \draw[densely dotted,thick] (EB) -- ++(-.5,0);
    \end{scope}

}

    \end{tikzpicture}
  \end{center}

  \begin{equation*}
    \oint\cd \E = 
    \textcolor{good}{\int_{\mathcal{R}}\cd \E} +
    \int_{\mathcal{L}}\cd \E +
    \int_{\mathcal{C}}\cd \E
    = 
    i2\pi \sum_\nu z_\nu
  \end{equation*}

  \vskip 1em
  Applying the residue theorem from complex analysis.

\end{frame}

% TODO commented out workshop 2021
\iffalse

\begin{frame}
  \frametitle{Numeric integration}
  \framesubtitle{Equilibrium}

  \begin{block}{Example of numeric integration of equilibrium contour}
    \begin{itemize}
      \item Ensure the lowest integrated energy is \emph{far} below the lowest
      eigenvalue of your system
    \end{itemize}
    
    \begin{center}
      \begin{tikzpicture}[scale=.8]
        \begin{axis}[width=14cm,height=8cm,name=circ,only marks,
          ymin=0,ymax=15.5,xmin=-31,xmax=1.5,
          xtick={-28,-24,-20,-16,-12,-8,-4},
          extra x ticks={0.25},extra x tick label={$\mu$},
          xlabel={Real Energy [eV]},
          ylabel={Imaginary Energy [eV]}]
          \addplot table {../data/EQ_circle.dat};
          \addplot table {../data/EQ_fermi.dat};
          \addplot table {../data/EQ_pole.dat};
          \draw[densely dashed,green!50!black,very thick] (axis cs:-30,0.1) --
          (axis cs:2,0.1);
          \draw[->,>=latex] (axis cs:-.5,0) -- (axis cs:-8.75,2.2);
          \draw[->,>=latex] (axis cs:-.5,1.25) -- (axis cs:-8.75,10.75);
          \node[rotate=35] at (axis cs:-21.5,11.5) {Gauss-Legendre};
        \end{axis}
        \begin{axis}[width=6cm,height=5cm,only marks,
          at={($(circ.south)+(0,1cm)$)},anchor=south,
          xtick={-0.5,0,0.5},
          extra x ticks={0.25},extra x tick label={$\mu$},
          xmin=-.5,xmax=.75,ymin=0,ymax=1.3]
          \addplot table {../data/EQ_circle.dat};
          \addplot table {../data/EQ_fermi.dat};
          \addplot table {../data/EQ_pole.dat};
          \draw[<->] (axis cs:.3,0.568494) -- node[sloped,anchor=south] {$2\pi k_BT$} (axis cs:.3,0.406067);
          \draw[densely dashed,green!50!black,very thick] 
          (axis cs:-1,0.03) -- (axis cs:1,0.03);
          \node[anchor=south] at (axis cs:0.25,1) {Gauss-Fermi};
          \node[rotate=90,anchor=south] at (axis cs:0.25,.5) {Poles};
        \end{axis}
      \end{tikzpicture}

    \end{center}
  \end{block}

\end{frame}


\pgfplotsset{my c/.style={/tikz/color=#1, /tikz/fill=#1}}

\begin{frame}
  \frametitle{Numeric integration}
  \framesubtitle{Algorithms}

  \begin{block}{Quadrature methods}
    TranSiesta implements a wide range of quadrature methods
    \begin{itemize}[<+->]
      \item Newton-Cotes quadratures
      \item Gauss-Legendre
      \begin{itemize}[<.->]
        \item<+-> Even quadrature method ($-x_{-i} = x_i$)
        \item Opportunity to only integrate half of the interval
      \end{itemize}
      \item Continued fraction
      \item \dots
    \end{itemize}
  \end{block}

  \begin{center}
    \begin{tikzpicture}[scale=.8]
      \begin{axis}[width=17cm,height=8cm,only marks,
        ymin=0,ymax=20.5,xmin=-41,xmax=1.5,
        xtick={-36,-32,-28,-24,-20,-16,-12,-8,-4},
        xlabel={Real Energy [eV]},
        ylabel={Imaginary Energy [eV]}]
        \only<1>{
        \addplot+[my c=red!70!black] table {../data/siesta_30_simpson.dat};}
        \only<2>{
        \addplot+[my c=blue!70!black] table {../data/siesta_30_legendre.dat};}
        \only<3->{
        \addplot+[my c=green!70!black] table {../data/siesta_30_legendre_right.dat};}
      \end{axis}
    \end{tikzpicture}
  \end{center}
  
  \doicite{Papior \etal: \doi{10.1016/j.cpc.2016.09.022}}

\end{frame}


\begin{frame}
  \frametitle{Numeric integration}
  \framesubtitle{Non-equilibrium}

  \begin{block}{Example of numeric integration of non-equilibrium contour}
    The non-equilibrium density is conceptually much easier:
    \begin{equation*}
      \DN \propto n_{F,\idxE}(\E) - n_{F,\idxE'}(\E)
    \end{equation*}

    \begin{center}
      \begin{tikzpicture}[scale=.8]
        \begin{axis}[width=15cm,height=8cm,
          ymin=0,ymax=2,xmin=-1,xmax=1,gen/.style={only marks,opacity=.7},
          xlabel={Energy [eV]},
          ylabel={Weight}]
          \only<1>{
              %\addplot[gen,bad,domain=-.3:.3,samples=25] {1./(exp((x-0.25)/0.01) + 1)-
              %    1./(exp((x+0.25)/0.01) + 1)};
              \addplot[gen,good,domain=-.55:.55,samples=30] {1./(exp((x-0.5)/0.01)
                  + 1)- 1./(exp((x+0.5)/0.01) + 1)};
              %\draw[<->,bad,very thick] (axis cs:-.25, 0.3) -- (axis cs:.25,0.3) node[midway,above]
              %{$n_F(\E+0.25\,\mathrm{eV}) - n_F(\E-0.25\,\mathrm{eV})$};
              \draw[<->,good,very thick] (axis cs:-.5, 1.3) -- (axis cs:.5,1.3) node[midway,above]
              {$n_F(\E+0.5\,\mathrm{eV}) - n_F(\E-0.5\,\mathrm{eV})$};
              \node[below left] at (rel axis cs:.9,.9) {2 different $\mu$};
          }
          \only<2>{
              \addplot[gen,ok,domain=-.05:0.55,samples=25] {1./(exp((x-0.5)/0.01)+1)-1./(exp((x)/0.01)+1)};
              \addplot[gen,good,domain=-.55:.55,samples=30] {1./(exp((x-0.5)/0.01)+1)-1./(exp((x+0.5)/0.01)+1)};
              \addplot[gen,bad,domain=-.55:.05,samples=25] {1./(exp((x)/0.01)+1)- 1./(exp((x+0.5)/0.01)+1)};
              \draw[<->,very thick,bad] (axis cs:-.5, 0.6) -- (axis cs:0,0.6) node[midway,above] 
              {$n_F(\E) - n_F(\E-0.5\,\mathrm{eV})$};
              \draw[<->,very thick,ok] (axis cs:0, 0.2) -- (axis cs:0.5,0.2) node[midway,above]
              {$n_F(\E+0.5\,\mathrm{eV}) - n_F(\E)$};
              \draw[<->,very thick,good] (axis cs:-.5, 1.3) -- (axis cs:.5,1.3) node[midway,above]
              {$n_F(\E+0.5\,\mathrm{eV}) - n_F(\E-0.5\,\mathrm{eV})$};

              \node[below left] at (rel axis cs:.9,.9) {3 different $\mu$};
          }

        \end{axis}
      \end{tikzpicture}
    \end{center}
  \end{block}

\end{frame}


% Skip the last summation of results page
\endinput

\begin{frame}
  \frametitle{Non-equilibrium Green function}
  \footnotesize

  \vskip -2ex
  \begin{columns}

    \column{.5\textwidth}
    \begin{block}{Basic equations}
      \vskip-2ex
\begin{align*}
  \G_\kk(\E)&=\big[(\E+\im\eta)\SO_\kk - \HH_\kk - \sum_\idxE\SE_{\idxE,\kk}(\E-\mu_\idxE)\big]^{-1}
  \\
  \Scat_{\idxE,\kk}(\E) &=i\big(\SE_{\idxE,\kk}(\E-\mu_\idxE)-\SE^\dagger_{\idxE,\kk}(\E-\mu_\idxE)\big)
  \\
  \Spec_{\idxE,\kk}(\E) &=\G_\kk(\E)\Scat_{\idxE,\kk}(\E)\G_\kk^\dagger(\E)
  % \\
  % \DM & =\frac1{2\pi}
  % \iint_\BZ\dEBZ\cd \kk \dd\E\, \sum_\idxE\Spec_{\idxE,\kk}(\E) n_{F,\idxE}(\E)
  % \eikr,
\end{align*}
    \end{block}

    \column{.5\textwidth}
    \begin{block}{Local Density of States}
      \vskip-2ex
      \begin{align*}
        \rho_{\nu}(\E) &= -\frac1\pi\Im[\G(\E) \SO]_{\nu} = \sum_\idxE\rho_\nu^\idxE(\E) +
        \text{bound states}
        \\
        \rho_{\nu}^{\idxE}(\E) &= \frac1{2\pi}\Re[\Spec_\idxE(\E) \SO]_{\nu}
      \end{align*}
    \end{block}

  \end{columns}

  \begin{block}{Density matrix using non-equilibrium Green functions}
    \vskip -2ex
    \begin{align*}
      \DM &=\frac1{2\pi}
      \iint_\BZ\dEBZ\cd \kk \dd\E\, \sum_\idxE\Spec_{\idxE,\kk}(\E) n_{F,\idxE}(\E)
      \eikr
      \\
      \DM^\varsigma &=\frac i{2\pi}\iint_\BZ\dEBZ\cd \kk\dd\E%
      \left[\G_\kk  - \G^\dagger_\kk \right]\eikr n_{F,\varsigma}(\E)
      +
      \frac1{2\pi}\sum_{\idxE|\varsigma_\idxE\neq\varsigma}\iint_\BZ\dEBZ\cd \kk\dd\E\, %
      \G_\kk \Scat_{\idxE,\kk}
      \G^\dagger_\kk\eikr\big[n_{F,\varsigma_\idxE}(\E)-n_{F,\varsigma}(\E)\big]
      & \text{$\varsigma = \{\mu, k_BT\}$}
    \end{align*}
  \end{block}

  \begin{block}{Transmission}
    \vskip-2ex
    \begin{columns}
      
      \column{.45\linewidth}
      \begin{align*}
        \T_{\idxE\mto\idxE'}(\E) &=
        \Tr\big[\Scat_{\idxE'}(\E)\G(\E)\Scat_{\idxE}(\E)\G^\dagger(\E)\big]\quad\text{, for $\idxE\neq\idxE'$}
        \\
        \T_{\idxE}(\E) &\equiv\sum_{\idxE'\neq\idxE}\T_{\idxE\mto\idxE'} (\E)
        \\
        \RE_\idxE(\E) & %=\mathbf 1 -\T_\idxE (\E)
        = \mathbf 1 -\Big\{
        \im \Tr\big[(\G(\E)-\G^\dagger(\E))\Scat_\idxE(\E)\big]
        -\Tr[\Scat_\idxE(\E) \G(\E)\Scat_\idxE(\E)\G^\dagger(\E)]
        \Big\}
        \\
        I_{\idxE\mto\idxE'} &= \frac{e^2}{h}\iint\cd\E\dd\kk\, \T_{\idxE\mto\idxE'}(\E)[n_{F,\idxE}(\E) - n_{F,\idxE'}(\E)].
      \end{align*}

      \column{.41\linewidth}
      \vskip -2em
      \begin{equation*}
        \JJ_{\idxE,\nu\mu} = \frac e h \Im\big[
        \Spec_{\idxE,\nu\mu}(\HH_{\mu\nu} - \E\SO_{\mu\nu})
        -
        \Spec_{\idxE,\mu\nu}(\HH_{\nu\mu} - \E\SO_{\nu\mu})\big]
      \end{equation*}

    \end{columns}
  \end{block}
  
\end{frame}

\fi

%%% Local Variables:
%%% mode: latex
%%% TeX-master: "talk"
%%% End:
