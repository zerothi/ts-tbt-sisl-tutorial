
\section{Electrodes -- options}

\begin{framenologo}
  \frametitle{Electrodes -- options}
  \tableofcontents[currentsection]
\end{framenologo}


\begin{frame}[label=electrode]
  \frametitle{Electrode -- options}

  \only<1>{Define names of electrodes}%
  \only<2>{Define each electrode}%
  \only<3>{Assert that the electrode settings reflect the input\footnote{\emph{HINT}: Do \emph{NOT} specify
          \texttt{used-atoms} if you use \emph{ALL} atoms!}}%
  \only<4>{When reducing the used atoms, it becomes slightly more difficult}%
  \only<5>{Automatically detects semi-infinite lattice vectors}%
  \only<6>{Buffer atoms removes atoms from TBtrans/TranSiesta SCF cycle}%

  \begin{tikzpicture}[thick,left/.style={densely dotted,color=bad},
    right/.style={color=ok}]

    % Define the electrode and we are done
    \begin{scope}
      \matrix {
          \node[fdf] {\%block TS.Elecs}; \\
          \node[fdf,ind,lmark] (left) {Left}; \\
          \node[fdf,ind,lmark] (right) {Right}; \\
          \node[fdf,ind] {...}; \\
          \node[fdf] {\%endblock}; \\
      };
    \end{scope}
    
    \uncover<2->{
        \begin{scope}[xshift=4.5cm,yshift=1.5cm]
          \matrix {
              \node[fdf,rmark] (l) {\%block TS.Elec.Left}; \\
              \node[fdf,ind] (l-HS) {HS <TSHS-file>/<nc-file>}; \\
              \node[fdf,ind] {chem-pot <in TS.ChemPots>}; \\
              \alt<3>{
                  \node[fdf,ind,left] (l-pos) {elec-pos 1}; \\
              }{
                  \node[fdf,ind] (l-pos) {elec-pos 1}; \\
              }
              \alt<4-5>{
                  \node[fdf,ind,left] (l-semi) {semi-inf-dir -a2}; \\
              }{
                  \node[fdf,ind] (l-semi) {semi-inf-dir -a2}; \\
              }
              \alt<4,6>{
                  \node[fdf,ind,left] (l-used) {used-atoms 3}; \\
              }{
                  \node[fdf,ind] (l-used) {used-atoms 4}; \\
              }
              \node[fdf,ind] (l-bloch) {bloch 1 1 1}; \\
              \node[fdf,ind] (l-eta) {eta <energy>}; \\
              \node[fdf] {\%endblock};\\
          };
        \end{scope}
        \draw[->] (left) -- ++(1.5,0) to[out=0,in=180] (l);

        \begin{scope}[xshift=4.5cm,yshift=-3.5cm]
          \matrix {
              \node[fdf,rmark] (r) {\%block TS.Elec.Right}; \\
              \node[fdf,ind] (r-HS) {HS <TSHS-file>/<nc-file>}; \\
              \node[fdf,ind] {chem-pot <in TS.ChemPots>}; \\
              \alt<3>{
                  \node[fdf,ind,right] (r-pos) {elec-pos end -1}; \\
              }{
                  \node[fdf,ind] (r-pos) {elec-pos end -1}; \\
              }
              \alt<4-5>{
                  \node[fdf,ind,right] (r-semi) {semi-inf-dir +a2}; \\
              }{
                  \node[fdf,ind] (r-semi) {semi-inf-dir +a2}; \\
              }
              \alt<4>{
                  \node[fdf,ind,right] (r-used) {used-atoms 3}; \\
              }{
                  \node[fdf,ind] (r-used) {used-atoms 4}; \\
              }
              \node[fdf,ind] {...}; \\
              \node[fdf] {\%endblock};\\
          };
        \end{scope}

        \draw[->] (right) -- ++(1,0) to[out=0,in=180] (r);
        
    }

    \uncover<3->{
        \begin{scope}[xshift=12.75cm,yshift=3cm]
          \begin{scope}[scale=0.7,
            every node/.style={scale=0.7,
                inner sep=.01ex,outer sep=.01ex,
            }]
            \matrix {
                \node[fdf] {\%block Atomic...Species}; \\
                \node[fdf,ind] (1) {\texttt{0. 0. 0.0  1}}; \\
                \node[fdf,ind] (2) {\texttt{0. 0. 1.5  1}}; \\
                \node[fdf,ind] (3) {\texttt{0. 0. 3.0  1}}; \\
                \node[fdf,ind] (4) {\texttt{0. 0. 4.5  1}}; \\
                \node[fdf] {\%endblock};\\
                \node[fdf] {\%block LatticeVectors}; \\
                \node[fdf,ind] {\texttt{ 0. 10.  0.}}; \\
                \alt<5>{
                    \node[fdf,ind,rmark] (e-A2) {\texttt{ 0.  0.  6.}}; \\
                }{
                    \node[fdf,ind] (e-A2) {\texttt{ 0.  0.  6.}}; \\
                }
                \node[fdf,ind] {\texttt{10.  0.  0.}}; \\
                \node[fdf] {\%endblock};\\
            };
          \end{scope}


          \begin{scope}[yshift=-5cm,scale=0.7,
            every node/.style={scale=0.7,
                inner sep=.01ex,outer sep=.01ex,
            }]
            \matrix {
                \node[fdf] {\%block Atomic...Species}; \\
                \alt<6>{
                    \node[fdf,ind,gray] (L1) {\texttt{0. 0.  0.0  1}}; \\
                }{
                    \node[fdf,ind] (L1) {\texttt{0. 0.  0.0  1}}; \\
                }
                \node[fdf,ind] (L2) {\texttt{0. 0.  1.5  1}}; \\
                \node[fdf,ind] (L3) {\texttt{0. 0.  3.0  1}}; \\
                \node[fdf,ind] (L4) {\texttt{0. 0.  4.5  1}}; \\
                \node[fdf,ind] {\texttt{0. 0.  6.0  1}}; \\
                \node[fdf,ind] {\texttt{0. 0.  7.5  1}}; \\
                \node[fdf,ind] {\texttt{0. 0.  9.0  1}}; \\
                \node[fdf,ind] {\texttt{0. 0. 10.5  1}}; \\
                \node[fdf,ind] {\texttt{0. 0. 12.0  1}}; \\
                \node[fdf,ind] {\texttt{0. 0. 13.5  1}}; \\
                \node[fdf,ind] (R4) {\texttt{0. 0. 15.0  1}}; \\
                \node[fdf,ind] (R3) {\texttt{0. 0. 16.5  1}}; \\
                \node[fdf,ind] (R2) {\texttt{0. 0. 18.0  1}}; \\
                \node[fdf,ind] (R1) {\texttt{0. 0. 19.5  1}}; \\
                \node[fdf] {\%endblock};\\
                \node[fdf] {\%block LatticeVectors}; \\
                \node[fdf,ind] (d-A1) {\texttt{ 0.  0. 21.}}; \\
                \node[fdf,ind] {\texttt{ 0. 10.  0.}}; \\
                \node[fdf,ind] {\texttt{10.  0.  0.}}; \\
                \node[fdf] {\%endblock};\\
            };
          \end{scope}
        \end{scope}
  }


  \only<3>{
      \draw[decorate, decoration=brace]
      ($(4.south west)+(-0.1,0)$) -- ($(1.north west)+(-0.1,0)$) coordinate[midway] (full elec);
      \draw[->, s >] (l-used) to[out=0,in=180] (full elec);
      \draw[->, s >] (r-used) to[out=0,in=180] (full elec);

      \draw[->, s >,left] (l-pos) to[out=0,in=180] (L1);
      \draw[->, s >,right] (r-pos) to[out=0,in=180] (R1);

  }

  \only<4>{
      \draw[decorate, decoration=brace,left]
      ($(4.south west)+(-0.1,0)$) -- ($(2.north west)+(-0.1,0)$) coordinate[midway] (left elec);
      \draw[decorate, decoration=brace,right]
      ($(3.south west)+(-0.6,0)$) -- ($(1.north west)+(-0.6,0)$) coordinate[midway] (right elec);

      \draw[decorate, decoration=brace,left]
      ($(L3.south west)+(-0.1,0)$) -- ($(L1.north west)+(-0.1,0)$) coordinate[midway] (left dev);
      \draw[decorate, decoration=brace,right]
      ($(R1.south west)+(-0.1,0)$) -- ($(R3.north west)+(-0.1,0)$) coordinate[midway] (right dev);

      \draw[->,s >,left] (l-used) to[out=0,in=180] (left elec);
      \draw[->,s >,right] (r-used) to[out=0,in=180] (right elec);

      \draw[<->,s <>,left] (left elec) to[out=180,in=180] (left dev);
      \draw[<->,s <>,right] (right elec) to[out=180,in=180] (right dev);

  }

  \only<5>{
      \draw[->,left,s >] (l-semi) to[out=0,in=180] (e-A2);
      \draw[->,right,s >] (r-semi.east) -- ++(0.5,0) to[out=0,in=180] (e-A2);

      \draw[<->,s <>] (d-A1) to[out=180,in=180] (e-A2);

  }

  \only<6>{
      \begin{scope}[xshift=9cm,yshift=-4.5cm]
        \matrix {
          \node[fdf] {\%block TS.Atoms.Buffer}; \\
          \node[fdf,ind] (buf) {atom 1}; \\
          \node[fdf] {\%endblock}; \\
      };
      \end{scope}

      \draw[decorate, decoration=brace,left]
      ($(4.south west)+(-0.1,0)$) -- ($(2.north west)+(-0.1,0)$) coordinate[midway] (left ELEC);
      \draw[decorate, decoration=brace,right]
      ($(4.south west)+(-0.6,0)$) -- ($(1.north west)+(-0.6,0)$) coordinate[midway] (right ELEC);

      \draw[decorate, decoration=brace,left]
      ($(L4.south west)+(-0.1,0)$) -- ($(L2.north west)+(-0.1,0)$) coordinate[midway] (left DEV);
      \draw[decorate, decoration=brace,right]
      ($(R1.south west)+(-0.1,0)$) -- ($(R4.north west)+(-0.1,0)$) coordinate[midway] (right DEV);

      \draw[->,s >,left] (l-used) to[out=0,in=180] (left ELEC);
      \draw[->,s >,right] (r-used) to[out=0,in=180] (right ELEC);

      \draw[<->,s <>,left] (left ELEC) to[out=180,in=180] (left DEV);
      \draw[<->,s <>,right] (right ELEC) to[out=180,in=180] (right DEV);
      
  }
      
  \end{tikzpicture}

\end{frame}

\begin{frame}
  \frametitle{Electrode -- options}

  \begin{center}
    Let us go through that again!
  \end{center}

\end{frame}

\againframe{electrode}


% \begin{frame}
%   \frametitle{All together}

%   \begin{tikzpicture}

%     % write chemical potentials
%     \def\chem{left}
%     \def\bias{V/2}

%     \begin{scope}[block]

%       \uncover<1->{
%       \begin{scope}[xshift=5cm,yshift=-.5cm]
%         \matrix {
%             \node[fdf] {\%block TS.ChemPots}; \\
%             \node[fdf,ind,lmark] (chem) {\chem}; \\
%             \node[fdf,ind] {...}; \\
%             \node[fdf] {\%endblock TS.ChemPots}; \\
%         };
%       \end{scope}
%   }

%       \uncover<2->{
%       % write chemical potential
%       \begin{scope}[xshift=5.5cm,yshift=-4cm]
%         \matrix {
%             \node[fdf,rmark] (chem-b) {\%block TS.ChemPot.\chem}; \\
%             \node[fdf,ind] {mu \bias}; \\
%             \node[fdf,ind] {contour.eq}; \\
%             \node[fdf,iind] {begin}; \\
%             \node[fdf,iiind,lmark] (c-cont) {c-\chem}; \\
%             \node[fdf,iiind,lmark] (t-cont) {t-\chem}; \\
%             \node[fdf,iind] {end}; \\
%             \node[fdf] {\%endblock TS.ChemPot.\chem}; \\
%         };
%       \end{scope}}


%       \uncover<3->{
%       % define contours
%       \begin{scope}[xshift=.5cm,yshift=-4cm]
%         \matrix {
%             \node[fdf,rmark] (c-cont-b) {\%block TS.Contour.c-\chem}; \\
%             \node[fdf,ind] {part circle}; \\
%             \node[fdf,iind] {from <E-low> to <E-high>}; \\
%             \node[fdf,iiind] {points 40}; \\
%             \node[fdf,iiind] {method g-legendre}; \\
%             \node[fdf] {\%endblock TS.Contour.c-\chem}; \\
%         };
%       \end{scope}
%       \begin{scope}[xshift=10cm,yshift=-3.5cm]
%         \matrix {
%             \node[fdf,rmark] (t-cont-b) {\%block TS.Contour.t-\chem}; \\
%             \node[fdf,ind] {part tail}; \\
%             \node[fdf,iind] {from prev to inf}; \\
%             \node[fdf,iiind] {points 10}; \\
%             \node[fdf,iiind] {method g-fermi}; \\
%             \node[fdf] {\%endblock TS.Contour.t-\chem}; \\
%         };
%       \end{scope}}

%       \uncover<1->{
%       % Define the electrode and we are done
%       \def\elec{Left}
%       \begin{scope}
%         \matrix {
%             \node[fdf] {\%block TS.Elecs}; \\
%             \node[fdf,ind,lmark] (elec) {\elec}; \\
%             \node[fdf,ind] {...}; \\
%             \node[fdf] {\%endblock TS.Elecs}; \\
%         };
%       \end{scope}}

%       \uncover<2->{
%       \begin{scope}[xshift=10cm,yshift=-.25cm]
%         \matrix {
%             \node[fdf,rmark] (elec-b) {\%block TS.Elec.\elec}; \\
%             \node[fdf,ind] {HS <TSHS-file>}; \\
%             \node[fdf,ind,lmark] (elec-chem) {chem-pot \chem}; \\
%             \node[fdf,ind] {semi-inf-dir -a3}; \\
%             \node[fdf,ind] {elec-pos 1}; \\
%             \node[fdf] {\%endblock TS.Elec.\elec}; \\
%         };
%       \end{scope}
%   }
      
%       \begin{scope}[connect]
%         \uncover<2->{
%             \draw (chem) -- (chem-b);
%             \draw (elec) -- (elec-b);
%             \draw[<->] (chem) -- (elec-chem);
%         }
%         \uncover<3->{
%             \draw (c-cont.175) -- (c-cont-b.-7);
%             \draw (t-cont.15) -- (t-cont-b.-173);
%         }
%       \end{scope}

%     \end{scope}

%   \end{tikzpicture}

% \end{frame}



%%% Local Variables:
%%% mode: latex
%%% TeX-master: "talk"
%%% End:
